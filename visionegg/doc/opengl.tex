\chapter{OpenGL in vision research \label{opengl}}

OpenGL offers many advantages over traditional means of producing
stimuli for vision research:

\begin{itemize}

\item \strong{Flexible}\\
	Almost any conceivable stimulus, from sine wave gratings to
	complex 3D scenes, can be created.

\item \strong{Powerful}\\
	Specialized hardware allows realtime stimulus generation,
	including image movement, scaling, interpolation, warping, and
	complex 3D virtual environments.

\item \strong{Inexpensive}\\
	Consumer video cards cost very little compared with
	traditional visual stimulus hardware.

\item \strong{Suited for specialized tasks}\\
	Professional systems are available with a large (12 bit)
	dynamic range of luminaces to allow precise contrast control,
	hardware genlock synchronization, 16 bit hardware accumulation
	buffer, and other specialized features.

\end{itemize}

OpenGL is a standardized set of commands to direct a video card to
produce computer graphics.  The OpenGL standard refers only to this
set of commands; the details of how the commands are performed is left
to a specific implementation.  Because of this standardization, you
are free to pick the specific set of hardware, software, and operating
system which best fits your needs.

The performance of OpenGL based video cards has increased dramatically
in recent years. However, programming visual stimuli using OpenGL was
a specialized task and relatively few laboratories could afford the
programmers or the time necessary. The Vision Egg changes this by
providing several pre-defined stimulus types that work ``out of the
box''.

At the core of a video card is a graphics processor unit, or ``GPU'',
designed specifically for the stereotyped operations encountered in
computer graphics. Because they are limited to specific data types and
operations, GPUs are much faster than a general purpose processor at
the tasks they perform. The enormous computational power that these
cards bring is a fundamentally new tool in the creation of visual
stimuli for vision research.