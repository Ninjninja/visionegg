\chapter{theory of operation \label{theory}}

The Vision Egg can coordinate just about any task that is possible
with a computer.  This document is a high level overview of how the
Vision Egg works.

If you need to do triggering, data acquisition, precise temporal
control of stimuli, integration with other software or hardware, read
this document.

\section{Architecture}

The Vision Egg is more than a simple library for drawing visual
stimuli. The Vision Egg allows coordination with just about any
element of a computer.  It is an experimental time-keeper and command
delegator for calling functions with sophisticated and precise
temporal control.

To perform this role, the Vision Egg uses a paradigm in which
\strong{contollers} can modify any \strong{parameter}. A controller
is called at pre-defined moments in time and typically functions to
calculate the value of a variable that affects some aspect of the
experiment.

The \seealso{API reference} has more extensive documentation for the
\class{Presentation}, \class{Controller}, and
\class{ClassWithParameters} classes.

\section{Timing of stimulus display}

\section{Triggering}

\section{Data acquisition}

\section{Stimulus drawing}