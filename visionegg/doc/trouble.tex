\subsection{Frame skipping}

\subsubsection{How to tell if you're skipping frames}

Look for automatically generated warnings in your log file.

Call the go() method of Presentation with its argument
collect_timing_info set to a non-zero value. This will send a
histogram of all the frames drawn in the go loop to the log.  (Storing
this infomation log takes time and resources, and therefore may have a
negative impact on performance, which is why it is not enabled by
default.)

\subsubsection{Overview}
On a modern graphics card such as an nVidia GeForce 3, simple
stimuli such as sinusoidal gratings can be drawn thousands of times
per second. Considering that the fastest consumer displays update 200
times per second, it seems that every single frame should easily be
controlled by the Vision Egg. Indeed, in \emph{fast cycle mode}
(\see{theory}), many frames are usually drawn by the graphics card for
every one that is displayed to the screen unless the operating
system pre-empts the Vision Egg. \see{system_choices})

With some operating systems, the Vision Egg gets pre-empted quite
frequently. It may help to raise the priority of the Vision Egg
program. Doing this is platform specific, and the Vision Egg contains
code for linux systems. Also, make sure to stop any programs that
might cause unecessary disk access or other system resource usage.

In \emph{frame sync mode}, there is a different problem.  Becuase the
main control loop

\subsection{Cannot move window in Windows}

Set VisionEgg.Core.Parameters.check_events = 1.