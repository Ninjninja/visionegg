The following are my own impressions of how the Vision Egg performs on
various computer systems I have had the opportunity to use. They
represent my own experience, and are necessarily limited.  We are
always interested in your own experiences, and we'd love to hear
reports on the mailing list.

\subsection{Choosing an operating system}

Before we discuss hardware, a brief overview of operating systems.
The Vision Egg runs on SGI, Windows, Linux, and Mac OS X systems, but
each of these systems has important differences.

SGI machines are the premiere hardware on which the Vision Egg is
known to run. The most important features available on SGIs are 12-bit
luminance control, a realtime kernel scheduler called REACT that
eliminates the worry of dropped frames, and a hardware accumulation
buffer.

On commodity hardware, Windows (2000 Pro) is the best operating system
for running experiments. The kernel scheduler rarely, if ever, causes
frame-skipping, OpenGL video drivers are usually well-behaved, there's
a great program called PowerStrip which can create and fine tune video
modes, and the parallel port is available for inexpensive digital
input and output.

Linux running on x86 hardware offers lots of great features for the
programmer, but I have has some trouble with the kernel scheduler (in
2.4.12-smp and 2.4.17 linux) causing frame-skipping on the same
machines that do fine under Windows. Also, I have observed nVidia
video driver bugs at 200 Hz frame rates. However, ease of development
and the wealth of programming tools keep linux as the favored
operating system for development. The parallel port is also available
for inexpensive digital input and output. Unfortunately, I believe
that hardware accelerated OpenGL is not possible on the latest ATI
Radeon cards without commercial drivers.  I do not know the quality of
these drivers.

Mac OS X (version 10.1.4) is best used only for development and
testing because its scheduler frequently causes frame skipping.  The
Tcl/Tk based GUI module Tkinter is also full of bugs on Mac OS
X. Hopefully these problems will be addressed as Mac OS X matures.

As a side note, Mac OS Classic would be great as a Vision Egg
platform. Its is a ``cooperative multitasking'' operating system,
which would not be multitasking at all if the Vision Egg didn't yield
control to the scheduler. Now, if only someone would port the Python
OpenGL binding!

\subsection{Data acquisition}

WRITE PARALLEL PORT DAQ

To date, the only data acquisition hardware with an open source Vision
Egg interface is a MacAdios card.  

That code runs the card in a separate, dedicated data-acquisition
computer and communicates over the computer network via the
\class{DaqOverTCP} subclass of the \class{Daq} class. Using this
system, triggering data acquisition from the stimulus generation
signal is possible, as well as triggering stimulus generation from
external hardware.
